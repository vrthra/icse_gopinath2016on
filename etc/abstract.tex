% 1.State the problem
%2.Say why it’s an interestin gproblem
%3.Say what your solution achieves
%4.Say what follows from your solution
Although mutation analysis is considered the best way to evaluate the
effectiveness of a test suite, hefty computational cost often limits its use.
To address this problem, various mutation reduction strategies have been
proposed, all seeking to reduce the number of mutants
while maintaining the representativeness of an exhaustive mutation analysis.
While research has focused on the reduction achieved, the effectiveness of
these strategies in selecting representative mutants, and the limits in doing
so have not been investigated, either theoretically or empirically.

We investigate the practical limits to the effectiveness of mutation reduction
strategies, and provide a simple theoretical framework for thinking about the
absolute limits. Our results show that the limit in improvement of effectiveness
over random sampling for real-world open source programs is a mean of
only $13.078\%$.
Interestingly, there is no limit to the improvement that can be made by
addition of new mutation operators.

Given that this is the maximum that can be achieved with perfect advance
knowledge of mutation kills, what can be practically achieved may be much
worse. We conclude that more effort should be focused on enhancing mutations
than removing operators in the name of selective mutation for questionable
benefit.

