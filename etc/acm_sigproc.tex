\batchmode
% THIS IS SIGPROC-SP.TEX - VERSION 3.1
% WORKS WITH V3.2SP OF ACM_PROC_ARTICLE-SP.CLS
% APRIL 2009
%
% It is an example file showing how to use the 'acm_proc_article-sp.cls' V3.2SP
% LaTeX2e document class file for Conference Proceedings submissions.
% ----------------------------------------------------------------------------------------------------------------
% This .tex file (and associated .cls V3.2SP) *DOES NOT* produce:
%       1) The Permission Statement
%       2) The Conference (location) Info information
%       3) The Copyright Line with ACM data
%       4) Page numbering
% ---------------------------------------------------------------------------------------------------------------
% It is an example which *does* use the .bib file (from which the .bbl file
% is produced).
% REMEMBER HOWEVER: After having produced the .bbl file,
% and prior to final submission,
% you need to 'insert'  your .bbl file into your source .tex file so as to provide
% ONE 'self-contained' source file.
%
% Questions regarding SIGS should be sent to
% Adrienne Griscti ---> griscti@acm.org
%
% Questions/suggestions regarding the guidelines, .tex and .cls files, etc. to
% Gerald Murray ---> murray@hq.acm.org
%
% For tracking purposes - this is V3.1SP - APRIL 2009

\documentclass{sig-alternate}
\usepackage{float}
\usepackage{xspace}
\usepackage{Sweavel}
\usepackage{placeins}
\usepackage{comment}
\usepackage{cite}
\usepackage{float}
\usepackage{Sweavel}
\usepackage{placeins}
\usepackage{xspace}

\usepackage{algpseudocode}
\usepackage{algorithm}
\usepackage{algorithmicx}
\newcommand{\theHalgorithm}{\arabic{algorithm}}

% Define the title here
%\def\mytitle{Why We Need More Mutators, Not Less!}
\def\mytitle{On The Limits of Mutation Reduction Strategies}
% Say no to selective mutation
% Trying out more titles :), please suggest if there is a more catchy & relevant one.


\usepackage[bookmarks=false, urlcolor=blue, linkcolor=blue, citecolor=blue, colorlinks=true, pdftitle={\mytitle}, pdfauthor={Rahul Gopinath}]{hyperref}
%\usepackage[center]{caption}
%\linespread{2}
\usepackage{comment}

\usepackage{listings}
\usepackage{float}
\floatstyle{plain}
\newfloat{program}{t}{lop}
\floatname{program}{Figure}
\usepackage{color}

\definecolor{dkgreen}{rgb}{0,0.6,0}
\definecolor{gray}{rgb}{0.5,0.5,0.5}
\definecolor{mauve}{rgb}{0.58,0,0.82}

%\lstset{frame=tb,
%  language=Java,
%  aboveskip=3mm,
%  belowskip=3mm,
%  showstringspaces=false,
%  columns=flexible,
%  basicstyle={\small\ttfamily},
%  numbers=left,
%  numberstyle=\tiny\color{gray},
%  keywordstyle=\color{blue},
%  commentstyle=\color{dkgreen},
%  stringstyle=\color{mauve},
%  breaklines=true,
%  breakatwhitespace=true
%  tabsize=3
%}

%\newcounter{observation}
%\newcommand{\observation}[1]{\refstepcounter{observation}
 
%\begin{center}
%\noindent\Ovalbox{
%\begin{minipage}{0.93\columnwidth}
%\textbf{Observation \arabic{observation}:} #1
%\end{minipage}
%}
%\end{center}
%\vspace{-5pt}
%}

%\setlength{\intextsep}{0mm}
%\setlength{\belowcaptionskip}{1mm}

\usepackage{color}
\definecolor{lightgray}{gray}{0.75}

\newcommand\greybox[1]{%
  \vskip\baselineskip%
  \par\noindent\colorbox{lightgray}{%
    \begin{minipage}{0.99\columnwidth}#1\end{minipage}%
  }%
  \vskip\baselineskip%
}

\usepackage[center]{caption}
\usepackage{fancybox}

\usepackage[font=footnotesize]{subfig}

% IMPORTANT : Remove this if you get an error about copyright box
%\input{copyright}

\setcounter{topnumber}{2}
\setcounter{bottomnumber}{2}
\setcounter{totalnumber}{4}
\renewcommand{\topfraction}{0.85}
\renewcommand{\bottomfraction}{0.85}
\renewcommand{\textfraction}{0.15}
\renewcommand{\floatpagefraction}{0.7}

\newcommand{\ignore}[1]{}
 %------------begin Float Adjustment
%two column float page must be 90% full
%\renewcommand\dblfloatpagefraction{.90}
%two column top float can cover up to 80% of page
%\renewcommand\dbltopfraction{.80}
%float page must be 90% full
%\renewcommand\floatpagefraction{.90}
%top float can cover up to 80% of page
%\renewcommand\topfraction{.80}
%bottom float can cover up to 80% of page
%\renewcommand\bottomfraction{.80}
%at least 10% of a normal page must contain text
%\renewcommand\textfraction{.1}
%separation between floats and text
\setlength\dbltextfloatsep{9pt plus 5pt minus 3pt }
%separation between two column floats and text
\setlength\textfloatsep{4pt plus 2pt minus 1.5pt}

\begin{document}
\title{\mytitle}
\subtitle{}

%
% You need the command \numberofauthors to handle the 'placement
% and alignment' of the authors beneath the title.
%
% For aesthetic reasons, we recommend 'three authors at a time'
% i.e. three 'name/affiliation blocks' be placed beneath the title.
%
% NOTE: You are NOT restricted in how many 'rows' of
% "name/affiliations" may appear. We just ask that you restrict
% the number of 'columns' to three.
%
% Because of the available 'opening page real-estate'
% we ask you to refrain from putting more than six authors
% (two rows with three columns) beneath the article title.
% More than six makes the first-page appear very cluttered indeed.
%
% Use the \alignauthor commands to handle the names
% and affiliations for an 'aesthetic maximum' of six authors.
% Add names, affiliations, addresses for
% the seventh etc. author(s) as the argument for the
% \additionalauthors command.
% These 'additional authors' will be output/set for you
% without further effort on your part as the last section in
% the body of your article BEFORE References or any Appendices.

\numberofauthors{5}
\author{
Rahul Gopinath\\
       \affaddr{Oregon State University}\\
       \email{gopinath@eecs.orst.edu}
\alignauthor
Mohammad Amin Alipour\\
       \affaddr{Oregon State University}\\
       \email{alipour@eecs.orst.edu}
\alignauthor
Iftekhar Ahmed\\
       \affaddr{Oregon State University}\\
       \email{ahmedi@onid.orst.edu}
\and
Carlos Jensen\\
       \affaddr{Oregon State University}\\
       \email{cjensen@eecs.orst.edu}
\alignauthor
Alex Groce\\
       \affaddr{Oregon State University}\\
       \email{agroce@gmail.com}
}


\CopyrightYear{2016}
\setcopyright{acmcopyright}
\conferenceinfo{ICSE '16,}{May 14-22, 2016, Austin, TX, USA}
\isbn{978-1-4503-3900-1/16/05}\acmPrice{\$15.00}
\doi{http://dx.doi.org/10.1145/2884781.2884787}

\maketitle
\begin{abstract}
% 1.State the problem
%2.Say why it’s an interestin gproblem
%3.Say what your solution achieves
%4.Say what follows from your solution
Although mutation analysis is considered the best way to evaluate the
effectiveness of a test suite, hefty computational cost often limits its use.
To address this problem, various mutation reduction strategies have been
proposed, all seeking to reduce the number of mutants
while maintaining the representativeness of an exhaustive mutation analysis.
While research has focused on the reduction achieved, the effectiveness of
these strategies in selecting representative mutants, and the limits in doing
so have not been investigated, either theoretically or empirically.

We investigate the practical limits to the effectiveness of mutation reduction
strategies, and provide a simple theoretical framework for thinking about the
absolute limits. Our results show that the limit in improvement of effectiveness
over random sampling for real-world open source programs is a mean of
only $13.078\%$.
Interestingly, there is no limit to the improvement that can be made by
addition of new mutation operators.

Given that this is the maximum that can be achieved with perfect advance
knowledge of mutation kills, what can be practically achieved may be much
worse. We conclude that more effort should be focused on enhancing mutations
than removing operators in the name of selective mutation for questionable
benefit.


\end{abstract}

%This is from the 98 class, 2012 does not have the number.
\category{D.2.5}{Software Engineering}{Testing and Debugging}~{Testing Tools}

\terms{Measurement, Verification}

\keywords{software testing, statistical analysis, mutation analysis} % NOT required for Proceedings


\input{paper}
%
% The following two commands are all you need in the
% initial runs of your .tex file to
% produce the bibliography for the citations in your paper.
\bibliographystyle{abbrv}
\bibliography{paper}  % sigproc.bib is the name of the Bibliography in this case
\end{document}
